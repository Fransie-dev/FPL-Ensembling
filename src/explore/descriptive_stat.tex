% Options for packages loaded elsewhere
\PassOptionsToPackage{unicode}{hyperref}
\PassOptionsToPackage{hyphens}{url}
%
\documentclass[
]{article}
\usepackage{lmodern}
\usepackage{amssymb,amsmath}
\usepackage{ifxetex,ifluatex}
\ifnum 0\ifxetex 1\fi\ifluatex 1\fi=0 % if pdftex
  \usepackage[T1]{fontenc}
  \usepackage[utf8]{inputenc}
  \usepackage{textcomp} % provide euro and other symbols
\else % if luatex or xetex
  \usepackage{unicode-math}
  \defaultfontfeatures{Scale=MatchLowercase}
  \defaultfontfeatures[\rmfamily]{Ligatures=TeX,Scale=1}
\fi
% Use upquote if available, for straight quotes in verbatim environments
\IfFileExists{upquote.sty}{\usepackage{upquote}}{}
\IfFileExists{microtype.sty}{% use microtype if available
  \usepackage[]{microtype}
  \UseMicrotypeSet[protrusion]{basicmath} % disable protrusion for tt fonts
}{}
\makeatletter
\@ifundefined{KOMAClassName}{% if non-KOMA class
  \IfFileExists{parskip.sty}{%
    \usepackage{parskip}
  }{% else
    \setlength{\parindent}{0pt}
    \setlength{\parskip}{6pt plus 2pt minus 1pt}}
}{% if KOMA class
  \KOMAoptions{parskip=half}}
\makeatother
\usepackage{xcolor}
\IfFileExists{xurl.sty}{\usepackage{xurl}}{} % add URL line breaks if available
\IfFileExists{bookmark.sty}{\usepackage{bookmark}}{\usepackage{hyperref}}
\hypersetup{
  pdftitle={letssee},
  pdfauthor={Francois van Zyl},
  hidelinks,
  pdfcreator={LaTeX via pandoc}}
\urlstyle{same} % disable monospaced font for URLs
\usepackage[margin=1in]{geometry}
\usepackage{color}
\usepackage{fancyvrb}
\newcommand{\VerbBar}{|}
\newcommand{\VERB}{\Verb[commandchars=\\\{\}]}
\DefineVerbatimEnvironment{Highlighting}{Verbatim}{commandchars=\\\{\}}
% Add ',fontsize=\small' for more characters per line
\usepackage{framed}
\definecolor{shadecolor}{RGB}{248,248,248}
\newenvironment{Shaded}{\begin{snugshade}}{\end{snugshade}}
\newcommand{\AlertTok}[1]{\textcolor[rgb]{0.94,0.16,0.16}{#1}}
\newcommand{\AnnotationTok}[1]{\textcolor[rgb]{0.56,0.35,0.01}{\textbf{\textit{#1}}}}
\newcommand{\AttributeTok}[1]{\textcolor[rgb]{0.77,0.63,0.00}{#1}}
\newcommand{\BaseNTok}[1]{\textcolor[rgb]{0.00,0.00,0.81}{#1}}
\newcommand{\BuiltInTok}[1]{#1}
\newcommand{\CharTok}[1]{\textcolor[rgb]{0.31,0.60,0.02}{#1}}
\newcommand{\CommentTok}[1]{\textcolor[rgb]{0.56,0.35,0.01}{\textit{#1}}}
\newcommand{\CommentVarTok}[1]{\textcolor[rgb]{0.56,0.35,0.01}{\textbf{\textit{#1}}}}
\newcommand{\ConstantTok}[1]{\textcolor[rgb]{0.00,0.00,0.00}{#1}}
\newcommand{\ControlFlowTok}[1]{\textcolor[rgb]{0.13,0.29,0.53}{\textbf{#1}}}
\newcommand{\DataTypeTok}[1]{\textcolor[rgb]{0.13,0.29,0.53}{#1}}
\newcommand{\DecValTok}[1]{\textcolor[rgb]{0.00,0.00,0.81}{#1}}
\newcommand{\DocumentationTok}[1]{\textcolor[rgb]{0.56,0.35,0.01}{\textbf{\textit{#1}}}}
\newcommand{\ErrorTok}[1]{\textcolor[rgb]{0.64,0.00,0.00}{\textbf{#1}}}
\newcommand{\ExtensionTok}[1]{#1}
\newcommand{\FloatTok}[1]{\textcolor[rgb]{0.00,0.00,0.81}{#1}}
\newcommand{\FunctionTok}[1]{\textcolor[rgb]{0.00,0.00,0.00}{#1}}
\newcommand{\ImportTok}[1]{#1}
\newcommand{\InformationTok}[1]{\textcolor[rgb]{0.56,0.35,0.01}{\textbf{\textit{#1}}}}
\newcommand{\KeywordTok}[1]{\textcolor[rgb]{0.13,0.29,0.53}{\textbf{#1}}}
\newcommand{\NormalTok}[1]{#1}
\newcommand{\OperatorTok}[1]{\textcolor[rgb]{0.81,0.36,0.00}{\textbf{#1}}}
\newcommand{\OtherTok}[1]{\textcolor[rgb]{0.56,0.35,0.01}{#1}}
\newcommand{\PreprocessorTok}[1]{\textcolor[rgb]{0.56,0.35,0.01}{\textit{#1}}}
\newcommand{\RegionMarkerTok}[1]{#1}
\newcommand{\SpecialCharTok}[1]{\textcolor[rgb]{0.00,0.00,0.00}{#1}}
\newcommand{\SpecialStringTok}[1]{\textcolor[rgb]{0.31,0.60,0.02}{#1}}
\newcommand{\StringTok}[1]{\textcolor[rgb]{0.31,0.60,0.02}{#1}}
\newcommand{\VariableTok}[1]{\textcolor[rgb]{0.00,0.00,0.00}{#1}}
\newcommand{\VerbatimStringTok}[1]{\textcolor[rgb]{0.31,0.60,0.02}{#1}}
\newcommand{\WarningTok}[1]{\textcolor[rgb]{0.56,0.35,0.01}{\textbf{\textit{#1}}}}
\usepackage{graphicx,grffile}
\makeatletter
\def\maxwidth{\ifdim\Gin@nat@width>\linewidth\linewidth\else\Gin@nat@width\fi}
\def\maxheight{\ifdim\Gin@nat@height>\textheight\textheight\else\Gin@nat@height\fi}
\makeatother
% Scale images if necessary, so that they will not overflow the page
% margins by default, and it is still possible to overwrite the defaults
% using explicit options in \includegraphics[width, height, ...]{}
\setkeys{Gin}{width=\maxwidth,height=\maxheight,keepaspectratio}
% Set default figure placement to htbp
\makeatletter
\def\fps@figure{htbp}
\makeatother
\setlength{\emergencystretch}{3em} % prevent overfull lines
\providecommand{\tightlist}{%
  \setlength{\itemsep}{0pt}\setlength{\parskip}{0pt}}
\setcounter{secnumdepth}{-\maxdimen} % remove section numbering
\usepackage{booktabs}
\usepackage{longtable}
\usepackage{array}
\usepackage{multirow}
\usepackage{wrapfig}
\usepackage{float}
\usepackage{colortbl}
\usepackage{pdflscape}
\usepackage{tabu}
\usepackage{threeparttable}
\usepackage{threeparttablex}
\usepackage[normalem]{ulem}
\usepackage{makecell}
\usepackage{xcolor}

\title{letssee}
\author{Francois van Zyl}
\date{13/09/2021}

\begin{document}
\maketitle

\begin{Shaded}
\begin{Highlighting}[]
\CommentTok{# Libraries ---------------------------------------------------------------}
\KeywordTok{library}\NormalTok{(ggplot2)}
\end{Highlighting}
\end{Shaded}

\begin{verbatim}
## Warning: package 'ggplot2' was built under R version 3.6.3
\end{verbatim}

\begin{Shaded}
\begin{Highlighting}[]
\KeywordTok{library}\NormalTok{(knitr)}
\end{Highlighting}
\end{Shaded}

\begin{verbatim}
## Warning: package 'knitr' was built under R version 4.0.2
\end{verbatim}

\begin{Shaded}
\begin{Highlighting}[]
\KeywordTok{library}\NormalTok{(dplyr)}
\end{Highlighting}
\end{Shaded}

\begin{verbatim}
## Warning: package 'dplyr' was built under R version 3.6.3
\end{verbatim}

\begin{verbatim}
## 
## Attaching package: 'dplyr'
\end{verbatim}

\begin{verbatim}
## The following objects are masked from 'package:stats':
## 
##     filter, lag
\end{verbatim}

\begin{verbatim}
## The following objects are masked from 'package:base':
## 
##     intersect, setdiff, setequal, union
\end{verbatim}

\begin{Shaded}
\begin{Highlighting}[]
\KeywordTok{library}\NormalTok{(kableExtra)}
\end{Highlighting}
\end{Shaded}

\begin{verbatim}
## Warning: package 'kableExtra' was built under R version 4.0.2
\end{verbatim}

\begin{verbatim}
## 
## Attaching package: 'kableExtra'
\end{verbatim}

\begin{verbatim}
## The following object is masked from 'package:dplyr':
## 
##     group_rows
\end{verbatim}

\begin{Shaded}
\begin{Highlighting}[]
\KeywordTok{library}\NormalTok{(autoEDA)}
\KeywordTok{library}\NormalTok{(rattle)}
\end{Highlighting}
\end{Shaded}

\begin{verbatim}
## Warning: package 'rattle' was built under R version 3.6.3
\end{verbatim}

\begin{verbatim}
## Loading required package: tibble
\end{verbatim}

\begin{verbatim}
## Warning: package 'tibble' was built under R version 3.6.3
\end{verbatim}

\begin{verbatim}
## Loading required package: bitops
\end{verbatim}

\begin{verbatim}
## Warning: package 'bitops' was built under R version 4.0.0
\end{verbatim}

\begin{verbatim}
## Rattle: A free graphical interface for data science with R.
## Version 5.4.0 Copyright (c) 2006-2020 Togaware Pty Ltd.
## Type 'rattle()' to shake, rattle, and roll your data.
\end{verbatim}

\begin{Shaded}
\begin{Highlighting}[]
\KeywordTok{library}\NormalTok{(rpart)}
\KeywordTok{library}\NormalTok{(rpart.plot)}
\end{Highlighting}
\end{Shaded}

\begin{verbatim}
## Warning: package 'rpart.plot' was built under R version 3.6.3
\end{verbatim}

\begin{Shaded}
\begin{Highlighting}[]
\KeywordTok{library}\NormalTok{(partykit)}
\end{Highlighting}
\end{Shaded}

\begin{verbatim}
## Warning: package 'partykit' was built under R version 3.6.3
\end{verbatim}

\begin{verbatim}
## Loading required package: grid
\end{verbatim}

\begin{verbatim}
## Loading required package: libcoin
\end{verbatim}

\begin{verbatim}
## Warning: package 'libcoin' was built under R version 4.0.2
\end{verbatim}

\begin{verbatim}
## Loading required package: mvtnorm
\end{verbatim}

\begin{verbatim}
## Warning: package 'mvtnorm' was built under R version 4.0.0
\end{verbatim}

\begin{Shaded}
\begin{Highlighting}[]
\KeywordTok{library}\NormalTok{(tidyrules)}
\end{Highlighting}
\end{Shaded}

\begin{verbatim}
## Warning: package 'tidyrules' was built under R version 3.6.3
\end{verbatim}

\begin{Shaded}
\begin{Highlighting}[]
\KeywordTok{library}\NormalTok{(papeR) }\CommentTok{# For formatting}
\end{Highlighting}
\end{Shaded}

\begin{verbatim}
## Warning: package 'papeR' was built under R version 3.6.3
\end{verbatim}

\begin{verbatim}
## Loading required package: car
\end{verbatim}

\begin{verbatim}
## Loading required package: carData
\end{verbatim}

\begin{verbatim}
## Warning: package 'carData' was built under R version 4.0.0
\end{verbatim}

\begin{verbatim}
## 
## Attaching package: 'car'
\end{verbatim}

\begin{verbatim}
## The following object is masked from 'package:dplyr':
## 
##     recode
\end{verbatim}

\begin{verbatim}
## Loading required package: xtable
\end{verbatim}

\begin{verbatim}
## Warning: package 'xtable' was built under R version 4.0.2
\end{verbatim}

\begin{verbatim}
## Registered S3 method overwritten by 'papeR':
##   method    from
##   Anova.lme car
\end{verbatim}

\begin{verbatim}
## 
## Attaching package: 'papeR'
\end{verbatim}

\begin{verbatim}
## The following objects are masked from 'package:dplyr':
## 
##     summarise, summarize
\end{verbatim}

\begin{verbatim}
## The following object is masked from 'package:utils':
## 
##     toLatex
\end{verbatim}

\begin{Shaded}
\begin{Highlighting}[]
\CommentTok{# Separate numerical + categorical ----------------------------------------}
\KeywordTok{setwd}\NormalTok{(}\StringTok{'C://Users//jd-vz//Desktop//Code//src//explore//'}\NormalTok{)}
\NormalTok{df <-}\StringTok{ }\KeywordTok{read.csv}\NormalTok{(}\DataTypeTok{file =} \StringTok{'C://Users//jd-vz//Desktop//Code//data//collected_fpl.csv'}\NormalTok{)}
\CommentTok{# df <- read.csv(file = 'C://Users//jd-vz//Desktop//Code//data//collected_us.csv')}
\NormalTok{num_feat <-}\StringTok{ }\NormalTok{df[}\KeywordTok{sapply}\NormalTok{(df, is.numeric)]}
\NormalTok{cat_feat <-}\StringTok{ }\NormalTok{df[,}\KeywordTok{colnames}\NormalTok{(df)[}\KeywordTok{grepl}\NormalTok{(}\StringTok{'factor|logical|character'}\NormalTok{,}\KeywordTok{sapply}\NormalTok{(df,class))]]}
\end{Highlighting}
\end{Shaded}

\begin{Shaded}
\begin{Highlighting}[]
\KeywordTok{length}\NormalTok{(}\KeywordTok{unique}\NormalTok{(df}\OperatorTok{$}\NormalTok{player_name)) }\CommentTok{# Unique players}
\end{Highlighting}
\end{Shaded}

\begin{verbatim}
## [1] 927
\end{verbatim}

\begin{Shaded}
\begin{Highlighting}[]
\KeywordTok{length}\NormalTok{(}\KeywordTok{unique}\NormalTok{(df}\OperatorTok{$}\NormalTok{team)) }\CommentTok{# Unique teams}
\end{Highlighting}
\end{Shaded}

\begin{verbatim}
## [1] 23
\end{verbatim}

\begin{Shaded}
\begin{Highlighting}[]
\KeywordTok{length}\NormalTok{(}\KeywordTok{unique}\NormalTok{(df}\OperatorTok{$}\NormalTok{kickoff_time)) }\CommentTok{# Unique dates}
\end{Highlighting}
\end{Shaded}

\begin{verbatim}
## [1] 250
\end{verbatim}

\begin{Shaded}
\begin{Highlighting}[]
\KeywordTok{length}\NormalTok{(}\KeywordTok{unique}\NormalTok{(df}\OperatorTok{$}\NormalTok{position)) }\CommentTok{# Unique positions}
\end{Highlighting}
\end{Shaded}

\begin{verbatim}
## [1] 4
\end{verbatim}

\begin{Shaded}
\begin{Highlighting}[]
\CommentTok{# These features could be interesting to investigate}
\KeywordTok{unique}\NormalTok{(df}\OperatorTok{$}\NormalTok{clean_sheets)}
\end{Highlighting}
\end{Shaded}

\begin{verbatim}
## [1] 0 1
\end{verbatim}

\begin{Shaded}
\begin{Highlighting}[]
\KeywordTok{unique}\NormalTok{(df}\OperatorTok{$}\NormalTok{clean_sheets)}
\end{Highlighting}
\end{Shaded}

\begin{verbatim}
## [1] 0 1
\end{verbatim}

\begin{Shaded}
\begin{Highlighting}[]
\KeywordTok{unique}\NormalTok{(df}\OperatorTok{$}\NormalTok{penalties_missed) }
\end{Highlighting}
\end{Shaded}

\begin{verbatim}
## [1] 0 1
\end{verbatim}

\begin{Shaded}
\begin{Highlighting}[]
\KeywordTok{unique}\NormalTok{(df}\OperatorTok{$}\NormalTok{penalties_saved) }\CommentTok{# This is not a binary feature}
\end{Highlighting}
\end{Shaded}

\begin{verbatim}
## [1] 0 1 2
\end{verbatim}

\begin{Shaded}
\begin{Highlighting}[]
\KeywordTok{unique}\NormalTok{(df}\OperatorTok{$}\NormalTok{yellow_cards)}
\end{Highlighting}
\end{Shaded}

\begin{verbatim}
## [1] 0 1
\end{verbatim}

\begin{Shaded}
\begin{Highlighting}[]
\KeywordTok{unique}\NormalTok{(df}\OperatorTok{$}\NormalTok{red_cards)}
\end{Highlighting}
\end{Shaded}

\begin{verbatim}
## [1] 0 1
\end{verbatim}

\begin{Shaded}
\begin{Highlighting}[]
\KeywordTok{unique}\NormalTok{(df}\OperatorTok{$}\NormalTok{was_home)}
\end{Highlighting}
\end{Shaded}

\begin{verbatim}
## [1] True  False
## Levels: False True
\end{verbatim}

\end{document}
